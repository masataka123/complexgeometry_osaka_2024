\documentclass[dvipdfmx,a4paper,12pt]{article}
\usepackage[utf8]{inputenc}
%\usepackage[dvipdfmx]{hyperref} %リンクを有効にする
\usepackage{url} %同上
\usepackage{amsmath,amssymb} %もちろん
\usepackage{amsfonts,amsthm,mathtools} %もちろん
\usepackage{braket,physics} %あると便利なやつ
\usepackage{bm} %ラプラシアンで使った
\usepackage[top=30truemm,bottom=30truemm,left=25truemm,right=25truemm]{geometry} %余白設定
\usepackage{latexsym} %ごくたまに必要になる
\renewcommand{\kanjifamilydefault}{\gtdefault}
\usepackage{otf} %宗教上の理由でmin10が嫌いなので


\usepackage[all]{xy}
\usepackage{amsthm,amsmath,amssymb,comment}
\usepackage{amsmath}    % \UTF{00E6}\UTF{0095}°\UTF{00E5}\UTF{00AD}\UTF{00A6}\UTF{00E7}\UTF{0094}¨
\usepackage{amssymb}  
\usepackage{color}
\usepackage{amscd}
\usepackage{amsthm}  
\usepackage{wrapfig}
\usepackage{comment}	
\usepackage{graphicx}
\usepackage{setspace}
\usepackage{pxrubrica}
\usepackage{enumitem}
\usepackage{mathrsfs} 
\usepackage[dvipdfmx]{hyperref}
\setstretch{1.2}

\newcommand{\mathsym}[1]{{}}
\newcommand{\unicode}[1]{{}}

\newcounter{mathematicapage}


%%%%%%%%% Theorem-like environment %%%%%%%%%%%
%
\theoremstyle{plain} %text of this environment is typesetted in italics
\newtheorem{theorem}{\indent\sc Theorem}[section]
\newtheorem{lemma}[theorem]{\indent\sc Lemma}
\newtheorem{corollary}[theorem]{\indent\sc Corollary}
\newtheorem{proposition}[theorem]{\indent\sc Proposition}
\newtheorem{claim}[theorem]{\indent\sc Claim}
\newtheorem{conjecture}[theorem]{\indent\sc Conjecture}
%
\theoremstyle{definition} %text of this environment is typesetted in roman letters
\newtheorem{definition}[theorem]{\indent\sc Definition}
\newtheorem{remark}[theorem]{\indent\sc Remark}
\newtheorem{example}[theorem]{\indent\sc Example}
\newtheorem{notation}[theorem]{\indent\sc Notation}
\newtheorem{assertion}[theorem]{\indent\sc Assertion}
\newtheorem{observation}[theorem]{\indent\sc Observation}
\newtheorem{problem}[theorem]{\indent\sc Problem}
\newtheorem{question}[theorem]{\indent\sc Question}
%
%If a theorem-like environment should not be numbered,
%add * after \newtheorem, and delete the counter option such as [theorem].
\newtheorem*{remark0}{\indent\sc Remark}
%
%%%%% Proof %%%%%
\renewcommand{\proofname}{\indent\sc Proof.}
%The following commands are available in the proof environment:
%\begin{proof}
%\end{proof}
%The end of a proof is marked with a square.
%%%%%%%%%%%%%%%%%%%%%%%%%%%%%%%%%%%%%%%%%

\begin{document}

\begin{center}
  {\Huge Workshop on Complex Geometry \\ in Osaka 2024}
 
  {\Large -Hodge theory and vanishing theorem-}
  %\vskip2mm{\LARGE Prospects and Open Problems \\ in Higher-dimensional Algebraic Geometry}
  \end{center}
  
\vskip5mm
\begin{flushleft}
{ Date: 13th--15th March 2024. (2024年3月13--15日)}


{Place: Lecture Room E404 in Graduate School of Science Building E in Osaka University (Toyonaka Campus).}
{(大阪大学理学部E404講義室(豊中キャンパス))}

\end{flushleft}


%\footnote{ホームページ: \texttt{https://sites.google.com/site/hisashikasuyamath/workshop-on-complex-geometry-in-osaka-2023?authuser=0}}
%\footnote{This conference is supported by Osaka City University Advanced Mathematical Institute: MEXT Joint Usage/Research Center on Mathematics and Theoretical Physics.}


\vskip8mm
\noindent{\Large \bf Program}
\vskip3mm

\noindent{\bf 3/13 (Wednesday)}
\vskip1mm
\noindent {\bf 13:30-14:30 }
{\bf Shouhei Ma (Tokyo Institute of Technology)}\\
Mixed Hodge structures of locally symmetric varieties
\vskip3mm

\noindent {\bf 14:45-15:45 } 
{\bf Yongpan Zou (The University of Tokyo)}\\
On the Kodaira-Saito Vanishing Theorem for Weakly Ample Divisors
\vskip3mm

\noindent {\bf16:15-17:15 } 
{\bf Yuta Watanabe (The University of Tokyo)}\\
Nakano-Nadel type, Bogomolov-Sommese type vanishing involving multiplier ideals

\vskip5mm
\noindent{\bf 3/14 (Thursday)}

\vskip1mm
\noindent {\bf 10:00-11:00 } 
{\bf Osamu Fujino (Kyoto University) }\\
Vanishing theorems for projective morphisms between complex analytic spaces 

\vskip3mm
\noindent {\bf 11:15-12:15 } 
{\bf Yota Shamoto (Waseda University) }\\
Stokes structure of difference modules

\vskip3mm
\noindent {\bf 14:15-15:15 } 
{\bf Takashi Ono (Osaka University)}\\
Wild harmonic bundles with skew-symmetric structure 

\vskip3mm
\noindent {\bf 15:45-16:45 } 
{\bf Shin-ichi Matsumura (Tohoku University)}\\
An injectivity theorem on snc compact Kahler spaces: an application of the theory of harmonic integrals on log-canonical centers via adjoint ideal sheaves

\vskip5mm
\noindent{\bf  3/15 (Friday)}
\vskip1mm
\noindent {\bf 10:00-11:00 } 
 {\bf Sheng Rao (Wuhan University)}\\
Geometry of logarithmic forms and deformations of complex structures

\vskip3mm
\noindent {\bf 11:15-12:15 } 
{\bf Takahiro Saito (Chuo University)}\\
mixed Hodge modules of normal crossing type on a smooth toric 
variety





%%%%%%%%%%%%%%%%%%%%%%%%%%%%%%%%%%%%
\begin{comment}

\begin{table}[htb]
\centering
 % \caption{スペック比較:罫線あり}
  \begin{tabular}{| c | | c | c | c |}  \hline
  Time  & 3/22(水) & 3/23(木)& 3/23(金)  \\ \hline 
     \begin{tabular}{c} GMT 6:00-7:00 \\ (JST 15:00-16:00)\end{tabular}
&Shin-ichi Matsumura & Jihun Yum&Seungjae Lee \\ \hline
      \begin{tabular}{c}   GMT 7:10-8:10 \\ (JST 16:10-17:10)  \end{tabular}
& Junchao Shentu & Hoseob Seo&\sout{ Juanyong Wang}\\ \hline
 \begin{tabular}{c}   GMT 8:40-9:40\\ (JST 17:40-18:40) \end{tabular}
&  Feng Hao& F\'elix Lequen& Lukas Braun \\ \hline
 \begin{tabular}{c}   GMT 11:30-12:30\\ (JST 20:30-21:30) \end{tabular}
&  Xiaojun Wu &  Olivier Thom &    \\ \hline
  \end{tabular}
\end{table}
\end{comment}
%%%%%%%%%%%%%%%%%%%%%%%%%%%%%%%%%

\vskip10mm
\noindent{\large \bf Organizers}
\begin{itemize}
  \setlength{\parskip}{0cm} 
  \setlength{\itemsep}{0cm}
\item Masataka Iwai (Osaka University)
\item Hisashi Kasuya (Osaka University)
  \end{itemize}
  
\noindent{\large \bf Supports}
\begin{itemize}
  \setlength{\parskip}{0cm} 
  \setlength{\itemsep}{0cm}
\item JSPS KAKENHI 22K13907 Grant-in-Aid for Early Career Scientists.
\item JSPS KAKENHI 19H01787 Grant-in-Aid for Scientific Research (B)
  \end{itemize}

\noindent{\large \bf Homepage}

We have posted various information on our website, including how to access to the conference room "Lecture Room E404".

\vskip3mm
Homepage Link: \url{https://masataka123.github.io/complexgeometry_osaka_2024/}

You can also read the QR code below:

\begin{figure}[htbp]
\begin{center}
 \includegraphics[height=50mm, width=50mm]{CGosaka2024.png}
\end{center}
\end{figure}



\newpage

\noindent{\Large \bf Abstract}
\vskip5mm

\noindent{\bf 3/13 (Wednesday)}
\vskip3mm

\noindent {\bf Shouhei Ma (Tokyo Institute of Technology)}\\
Mixed Hodge structures of locally symmetric varieties

\vskip3mm
I will talk about the mixed Hodge structures on the cohomology of locally symmetric varieties. 
In the middle degree, I relate the weight filtration to the Siegel operators for certain modular forms. This has an application to a classical problem on the Siegel operators. 
In the general degrees, I construct a spectral sequence which converges to the edge Hodge components in the Hodge triangle, and whose E1 page is expressed by some simple geometric invariants associated to the cusps. 
This already degenerates at E1 in a certain range. Applications contain a new proof of a classical result of Harder on the Eisenstein cohomology of Hilbert modular varieties. 
\vskip6mm

\noindent {\bf Yongpan Zou (The University of Tokyo)}\\
On the Kodaira-Saito Vanishing Theorem for Weakly Ample Divisors.

\vskip3mm
On a smooth projective variety, considering a reduced effective divisor that is weakly ample in the sense of cohomology, we introduce a Kodaira vanishing theorem for it. Our approach involves utilizing the Hodge ideal introduced by Mustata-Popa.
\vskip6mm

\noindent {\bf Yuta Watanabe (The University of Tokyo)}\\
Nakano-Nadel type, Bogomolov-Sommese type vanishing involving multiplier ideals

\vskip3mm
In this talk, we first obtain the Bogomolov-Sommese type vanishing theorem involving multiplier ideal sheaves for big line bundles. We investigate Griffiths and (dual) Nakano positivity for singular Hermitian metrics of holomorphic vector bundles and obtain various vanishing theorems.
\vskip6mm

\newpage 

\noindent{\bf 3/14 (Thursday)}
\vskip3mm

\noindent {\bf Osamu Fujino (Kyoto University)}\\
Vanishing theorems for projective morphisms between complex analytic spaces 

\vskip3mm
We discuss vanishing theorems for projective morphisms between complex analytics spaces and some related results. They will play a crucial role in the minimal model theory for projective morphisms of complex analytic spaces. Roughly speaking, we establish an ultimate generalization of Koll\'ar's package from the minimal model theoretic viewpoint.
\vskip6mm

\noindent {\bf Yota Shamoto (Waseda University) }\\
Stokes structure of difference modules

\vskip3mm
P. Deligne and B. Malgrange introduced the notion of Stokes filtered local systems as the Betti counterpart of meromorphic connection stalks at a point with irregular singularity in one variable. This talk introduces an analogous notion, i.e., the Betti counterpart for additive difference modules. We also explain some motivation from non-abelian Hodge theory for periodic monopoles.  
\vskip6mm

\noindent{\bf Takashi Ono (Osaka University)}\\
Wild harmonic bundles with skew-symmetric structure 

\vskip3mm
There is a one-on-one correspondence between a good wild harmonic bundle and a polystable good filtered Higgs bundle with vanishing Chern classes, a branch of Kobayashi-Hitchin correspondence.  In this talk, I want to show how the  Higgs bundle decomposes when the harmonic bundle has skew-symmetric pairing. Also, I want to show the Kobayashi-Hitchin correspondence with slew-symmetry.
\vskip6mm

\noindent {\bf Shin-ichi Matsumura (Tohoku University)}\\
An injectivity theorem on snc compact Kahler spaces:
an application of the theory of harmonic integrals on log-canonical centers via adjoint ideal sheaves


\vskip3mm
In this talk, I would like to discuss the injectivity theorem, an extension of the Kodaira vanishing theorem into the 'semi-positive' case from a complex analytic perspective.
Initially, I will review Kollar's formulation of injectivity for semi-ample line bundles in algebraic geometry and its generalization to semi-positive line bundles by Enoki in the complex analytic setting. 
Subsequently, my focus will shift to the injectivity theorem for log canonical (LC) pairs, established by Ambro and Fujino in the context of algebraic geometry using Hodge theory.
I would like to explain our proof for Fujino's conjecture, which asks for a complex analytic analog of Ambro and Fujino's result.
This is joint work with Tsz On Mario Chan and Young-Jun Choi (Pusan National University).
\vskip6mm


\newpage

\noindent{\bf 3/15  (Friday)}
\vskip3mm

\noindent{\bf Sheng Rao (Wuhan University)}\\
Geometry of logarithmic forms and deformations of complex structures

\vskip3mm
 We present a new method to solve certain dbar-equations for logarithmic differential forms by using harmonic integral theory for currents on Kahler manifolds. As applications, we generalize the result of Deligne about closedness of logarithmic forms, give geometric and simpler proofs of Deligne's degeneracy theorem for the logarithmic Hodge to de Rham spectral sequences at E1-level, as well as certain injectivity theorem on compact Kahler manifolds. Our method also plays an important role in Cao--Paun's recent works on the extension of pluricanonical sections and proof of Fujino's injectivity conjecture. 
     Furthermore, for a family of logarithmic deformations of complex structures on Kahler manifolds, we construct the extension for any logarithmic (n,q)-form on the central fiber and thus deduce the local stability of log Calabi--Yau structure by extending an iteration method to the logarithmic forms. Finally we prove the unobstructedness of the deformations of a log Calabi--Yau pair and a pair on a Calabi--Yau manifold by differential geometric method. Its projective case was originally obtained by Katzarkov--Kontsevich--Pantev in 2008. 
\vskip6mm


\noindent {\bf Takahiro Saito (Chuo University)}\\
mixed Hodge modules of normal crossing type on a smooth toric variety

\vskip3mm
In general, mixed Hodge modules (MHMs) are complicated and difficult to 
deal with, while the general theory is well established.
However, if the underlying D-module of a MHM on Cn satisfies the 
condition: ``of normal crossing type",
it can be expressed in a linear algebraic way.
This is a generalization to MHM of the well-known ``can-var description" 
of perverse sheaves on C.
As an application, we can give a natural definition of ``the Fourier-Laplace transform of a MHM of normal crossing type".

In the first half, I will introduce these facts.
In the second half, I will talk on the recent progress on the 
generalization of them to MHMs on smooth toric varieties and the 
application to the ``Hodge structure" of the Fukaya category of the 
plumbed manifolds of a Dynkin diagram (in collaboration with Tatsuki 
Kuwagaki).
\vskip6mm

\end{document}